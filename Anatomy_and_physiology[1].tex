
\documentclass[12pt]{article}

\begin{document}
\tableofcontents
\clearpage
\section{An Introduction of human Body}
\subsection{Anatomy and Physiology Defined}
Physiology is the study of how the body works, while anatomy studies how those parts relate to one another.Histology, gross anatomy, systemic anatomy, regional anatomy, surface anatomy, radiographic anatomy, and pathological anatomy are some of the areas of anatomy. Other branches include embryology, developmental biology, cell biology, and embryology.
Molecular physiology, neurophysiology, endocrinology, cardiovascular physiology, immunology, respiratory physiology, renal physiology, exercise physiology, and pathophysiology are some of the subspecialties of physiology.
\subsection{Structural Organization and Body Systems}
There are six structural layers in the human body: chemical, cellular, tissue, organ, system, and organisms. 
The smallest living entities in the human body, cells are the basic structural and functional living units of an organism. 
Tissues are collections of cells and the substances that surround them that cooperate to carry out a certain purpose Organs are made up of two or more different tissue types, and they typically have recognised shapes and specific functions.  Systems are made up of organs that are connected and perform the same task.
\subsection{Characterstics of the living Human Organism}
All living things engage in specific processes that set them apart from nonliving objects.Metabolism,responsiveness, movement, growth, differentiation, and reproduction are among the life processes in humans.
\subsection{Basic Anatomical Terminology}
When describing any part of the body, it is assumed that the subject is in the anatomical position, which involves the person standing straight in front of the observer with the head level and the eyes looking directly ahead. The upper limbs are at the sides with the palms facing forward while the feet are flat on the ground and pointed forward. Prone means facing down, whereas supine means facing up. Regional names are labels given to particular body regions. The head, neck, trunk, upper limbs, and lower limbs are considered to be the fundamental regions. Anatomical names and matching common names for particular body parts are used within the areas.Examples are the thoracic (chest), nasal, and carpal spine (wrist)In order to visualise internal anatomy, planes are fictitious flat surfaces that are utilised to divide the body or organsCuts performed along a plane to the body or its organs are called sections. They include transverse, frontal, and sagittal sections and are referred to by the plane along which the cut is made Three smaller cavities make up the thoracic cavity: a pericardial cavity, which houses the heart, and two pleural cavities, each of which houses a lung.The superior abdominal cavity and the inferior pelvic cavity are the two compartments of the abdominopelvic cavityThe organs found inside the thoracic and abdominal cavities are protected by serous membranes that border their walls.
\subsection{Aging and Homeostasis}
Changes in structure and function that can be seen as a result of ageing include: 
it is susceptible to sickness and stress.All of the body's systems undergo aging-related issues.
\subsection{Medical imaging}
Techniques and processes used to produce images of the human body are referred to as medical imaging. They enable the imaging of interior structures for the diagnosis of aberrant anatomy and physiologic abnormalities.
\subsection{Homeostasis}
Homeostasis is the state of steady internal, physical, and chemical conditions maintained by living systems. It is the condition of optimal functioning for the organism and includes many variables, such as body temperature, fluid balance etc..Homeostasis is controlled by the neurological and endocrine systems working together or independently. The nervous system notices when the body changes and delivers nerve impulses to counteract alterations under controlled circumstances. Hormone secretion by the endocrine system controls homeostasis. 
5. Feedback systems consist of the following three elements: Receptors track changes in a controlled environment and relay information to a control centre (affer- ent pathway).The control centre establishes the value (set point) at which a controlled condition should be maintained, assesses the information it gets from receptors (efferent route), and generates output commands as necessary.Effectors provide a response (effect) that changes the controlled situation after receiving output from the control centre.































\end{document}